% In this file you should put the actual content of the blueprint.
% It will be used both by the web and the print version.
% It should *not* include the \begin{document}
%
% If you want to split the blueprint content into several files then
% the current file can be a simple sequence of \input. Otherwise It
% can start with a \section or \chapter for instance.

\title{Bachman-Landau notation and the Master theorem for divide-and-conquer recurrences}

The primary goal of this project is to formalize some results from computational 
complexity theory in Lean 4:
\begin{itemize}
\item the Bachman-Landau family of notations, e.g. big O and other closely related notations,
\item properties of the Bachman-Landau notations and relations between them,
\item the Master theorem for divide-and-conquer recurrences,
\item and, if time permits, the Akra-Bazzi method which solves a more general class of recurrences
\end{itemize}

\section{Bachman-Landau notation}

Bachman-Landau family of notations is the name of a few closely related notations used in 
algorithm analysis. The most famous of them is the so-called big-O notation. While
most formulations are defined on functions from naturals or reals to reals, we define 
them more generally on functions from a linearly ordered commutative ring to a
linearly ordered field. In the rest of this page, we let $R$ be a linearly ordered 
commutative ring and $F$ be a linearly ordered field. We will also use symbols $f$ and 
$g$ for functions $R \to F$.

\subsection{Asymptotic bounds and domination}

\begin{definition}
    \label{def:asymp_bounded_above}
    \lean{AsymptoticallyBoundedAboveBy}
    \leanok
    $f(x)$ is asymptotically bounded above by $g(x)$ if there exist $x_0$ and $k > 0$ 
    such that $|f(x)| \le k*|g(x)|$ for all $x \ge x_0$.
\end{definition}

\begin{definition}(Big O notation)
    \label{def:big_o}
    \lean{O}
    \leanok
    $f(x) \in O(g(x))$ if it is asymptotically bounded above by $g(x)$.
\end{definition}

\begin{definition}
    \label{def:asymp_bounded_below}
    \lean{AsymptoticallyBoundedBelowBy}
    \leanok
    $f(x)$ is asymptotically bounded below by $g(x)$ if there exist $x_0$ and $k > 0$ 
    such that $|f(x)| \le k*|g(x)|$ for all $x \ge x_0$.
\end{definition}

\begin{definition}(Big Omega notation)
    \label{def:big_omega}
    \lean{Ω}
    \leanok
    $f(x) \in \Omega(g(x))$ if it is asymptotically bounded below by $g(x)$.
\end{definition}

\begin{definition}
    \label{def:asymp_bounded}
    \lean{AsymptoticallyBoundedBy}
    \leanok
    $f(x)$ is asymptotically bounded by $g(x)$ if there exist $x_0$, $k\_1 > 0$ 
    and $k\_2$ such that $k\_1*|g(x)| \le |f(x)| \le k\_2*|g(x)|$ for all $x \ge x_0$.
\end{definition}

\begin{definition}(Big Theta notation)
    \label{def:big_theta}
    \lean{Θ}
    \leanok
    $f(x) \in \Theta(g(x))$ if it is asymptotically bounded by $g(x)$. 
\end{definition}

\begin{definition}
    \label{def:asymp_dominated}
    \lean{AsymptoticallyDominatedBy}
    \leanok
    Let $f$ and $g$ be functions $R \to F$. $f(x)$ is asymptotically dominated by $g(x)$ if 
    for all $k > 0$ there exists $x_0$ such that $|f(x)| \le k*|g(x)|$ for all $x \ge x_0$.
\end{definition}

\begin{definition}(Small O notation)
    \label{def:small_o}
    \lean{o}
    \leanok
    $f(x) \in o(g(x))$ if it is asymptotically dominated by $g(x)$.
\end{definition}

\begin{definition}
    \label{def:asymp_dominates}
    \lean{AsymptoticallyDominates}
    \leanok
    $f(x)$ is asymptotically dominates $g(x)$ if for all $k > 0$ there exists $x_0$ 
    such that $|f(x)| \ge k*|g(x)|$ for all $x \ge x_0$.
\end{definition}

\begin{definition}(Small Omega notation)
    \label{def:small_omega}
    \lean{ω}
    \leanok
    $f(x) \in \omega(g(x))$ if it asymptotically dominates $g(x)$.
\end{definition}

%Goals (to be updated as necessary)
%\begin{itemize}
%\item definitions of notations $O(f(x))$, $\Omega(f(x))$, $\Theta(f(x))$, $o(f(x))$ and $\omega(f(x))$,
%\item basic relations between notations (e.g. $o(f(x)) \implies O(f(x))$),
%\item equivalence relations on asymptotically equivalent instances of each kind of notation,
%\item equivalences of notations with their limit definitions,
%\item \textbf{optional:} show properties of certain classes of instances - notably polynomials and exponentials,
%\item add accompanying examples
%\end{itemize}

\subsection{Limit definitions}

TODO

\subsection{Relations between asymptotic sets}

\begin{theorem}
    \label{thm:asymp_dominated_implies_bounded_above}
    \lean{asymp_dominated_implies_bounded_above}
    \leanok
    \uses{def:asymp_dominated, def:asymp_bounded_above}
    If $f(x)$ is dominated by $g(x)$, then it's also bounded above by $g(x)$.
\end{theorem}

\begin{proof}
    \leanok 
    The definitions of $f(x)$ being dominated and bounded above by $g(x)$ only differ
    in the quantifier before $k$ at the very start (universal for the former, existential
    for the latter), so it suffices to use any positive value for $k$. We use $1$. 
    The desired result then follows directly.
\end{proof}

\begin{theorem}
    \label{thm:small_o_implies_big_o}
    \lean{o_implies_O}
    \leanok
    \uses{def:small_o, def:big_o}
    If $f(x) \in o(g(x))$, then $f(x) \in O(f(x))$.
\end{theorem}

\begin{proof}
    \leanok
    \uses{thm:asymp_dominated_implies_bounded_above}
    Since $o(g(x))$ and $O(f(x))$, we can simply use Theorem 
    \ref{thm:asymp_dominated_implies_bounded_above}.
\end{proof}

\begin{theorem}
    \label{thm:asymp_dominates_implies_bounded_below}
    \lean{asymp_dominates_implies_bounded_below}
    \leanok
    \uses{def:asymp_dominates, def:asymp_bounded_below}
    If $f(x)$ dominates $g(x)$, then it's bounded below by $g(x)$.
\end{theorem}

\begin{proof}
    \leanok
    The proof is completely analogous to the proof of Theorem 
    \ref{thm:asymp_dominated_implies_bounded_above}, again using the value $1$ for $k$.
\end{proof}

\begin{theorem}
    \label{thm:small_omega_implies_big_omega}
    \lean{ω_implies_Ω}
    \leanok
    \uses{def:small_omega, def:big_omega}
    If $f(x) \in \omega(g(x))$, then $f(x) \in \Omega(g(x))$.
\end{theorem}

\begin{proof}
    \leanok
    \uses{thm:asymp_dominates_implies_bounded_below}
    The proof is a simple application of Theorem 
    \ref{thm:asymp_dominates_implies_bounded_below}.
\end{proof}

\begin{theorem}
    \label{thm:asymp_bounded_above_and_below_equiv_bounded}
    \lean{asymp_bounded_above_and_below_equiv_bounded}
    \leanok
    \uses{def:asymp_bounded_above, def:asymp_bounded_below, def:asymp_bounded}
    If $f(x)$ is bounded above and below by $g(x)$, then it's bounded by $g(x)$.
\end{theorem}

\begin{proof}
    \leanok
    TODO
\end{proof}

\begin{theorem}
    \label{thm:big_o_and_omega_equiv_theta}
    \lean{O_and_Ω_equiv_Θ}
    \leanok
    \uses{def:big_o, def:big_omega, def:big_theta}
    $f(x) \in O(g(x))$ and $f(x) \in \Omega(g(x))$ if and only if $f(x) \in \Theta(g(x))$.
\end{theorem}

\begin{proof}
    \leanok
    \uses{thm:asymp_bounded_above_and_below_equiv_bounded}
    Similarly to previous proofs, the proof is a direct application of Theorem 
    \ref{thm:asymp_bounded_above_and_below_equiv_bounded}.
\end{proof}


\section{The Master theorem}

Analyzing the time complexity of algorithms, especially recursive ones, is more often 
than not a non-trivial task. For a recursive algorithm, its time complexity can be 
written as a recurrence formula, which is generally not easy, sometimes even impossible 
to solve with a closed formula. In some cases, though, it turns out that we can find 
asymptotic bounds of the solution, despite not being able to necessarily find the 
precise solution to the recurrence.

Divide-and-conquer algorithms, i.e. algorithms that work by recursively splitting the input 
problem into similarly-sized subproblems have especially nice recurrence formulas which can be 
asymptotically bounded with a simple formula that is known as the Master theorem.

\begin{theorem}[Master theorem for divide-and-conquer recurrences]
TODO
\end{theorem}
