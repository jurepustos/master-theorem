% In this file you should put the actual content of the blueprint.
% It will be used both by the web and the print version.
% It should *not* include the \begin{document}
%
% If you want to split the blueprint content into several files then
% the current file can be a simple sequence of \input. Otherwise It
% can start with a \section or \chapter for instance.

\title{Bachman-Landau notation and the Master theorem for divide-and-conquer recurrences}

The primary goal of this project is to formalize some results from computational 
complexity theory in Lean 4:
\begin{itemize}
\item the Bachman-Landau family of notations, e.g. big O and other closely related notations,
\item properties of the Bachman-Landau notations and relations between them,
\item the Master theorem for divide-and-conquer recurrences,
\item and, if time permits, the Akra-Bazzi method which solves a more general class of recurrences
\end{itemize}

\section{Bachman-Landau notation}

Bachman-Landau family of notations is the name of a few closely related notations used in 
algorithm analysis. The most famous of them is the so-called big-O notation. While
most formulations are defined on functions from naturals or reals to reals, we define 
them more generally on functions from a linearly ordered commutative ring to a
linearly ordered field. In the rest of the formulation and blueprint, we let $R$ be
a linearly ordered commutative ring and $F$ be a linearly ordered field.

\subsection{Asymptotic bounds and domination}

\begin{definition}(Big O notation)
\lean{MasterTheorem.BachmanLandauNotation.O}
\leanok
    Let $f$ and $g$ be functions $R \to F$. Then $f(x) \in O(g(x))$ if there exist
    $x_0$ and $k > 0$ such that $|f(x)| \le k*|g(x)|$ for all $x \ge x_0$. We say that
    $f(x)$ is asymptotically bounded above by $g(x)$.
\end{definition}

\begin{definition}(Big Omega notation)
\lean{MasterTheorem.BachmanLandauNotation.\Omega}
\leanok
    Let $f$ and $g$ be functions $R \to F$. Then $f(x) \in \Omega(g(x))$ if there exist
    $x_0$ and $k > 0$ such that $|f(x)| \ge k*|g(x)|$ for all $x \ge x_0$. We say that
    $f(x)$ is asymptotically bounded below by $g(x)$. We say that $f(x)$ is asymptotically 
    bounded below by $g(x)$.
\end{definition}

\begin{definition}(Big Theta notation)
\lean{MasterTheorem.BachmanLandauNotation.\Theta}
\leanok
    Let $f$ and $g$ be functions $R \to F$. Then $f(x) \in \Theta(g(x))$ if there exist
    $x_0$, $k\_1 > 0$ and $\k\_2$ such that $k\_1*|g(x)| \le |f(x)| \le k\_2*|g(x)|$ 
    for all $x \ge x_0$. We say that $f(x)$ is asymptotically bounded below by $g(x)$. 
    We say that $f(x)$ is asymptotically bounded by $g(x)$.
\end{definition}

\begin{definition}(Small O notation)
\lean{MasterTheorem.BachmanLandauNotation.o}
\leanok
    Let $f$ and $g$ be functions $R \to F$. Then $f(x) \in o(g(x))$ if for all
    $k > 0$ there exists $x_0$ such that $|f(x)| \le k*|g(x)|$ for all $x \ge x_0$. 
    We say that $f(x)$ is asymptotically bounded below by $g(x)$. We say that $f(x)$ 
    is asymptotically dominated by $g(x)$.
\end{definition}

\begin{definition}(Small Omega notation)
\lean{MasterTheorem.BachmanLandauNotation.\omega}
\leanok
    Let $f$ and $g$ be functions $R \to F$. Then $f(x) \in \omega(g(x))$ if for all
    $k > 0$ there exists $x_0$ such that $|f(x)| \ge k*|g(x)|$ for all $x \ge x_0$. 
    We say that $f(x)$ is asymptotically bounded below by $g(x)$. We say that $f(x)$ 
    asymptotically dominates $g(x)$.
\end{definition}

Goals (to be updated as necessary)
\begin{itemize}
\item definitions of notations $O(f(x))$, $\Omega(f(x))$, $\Theta(f(x))$, $o(f(x))$ and $\omega(f(x))$,
\item basic relations between notations (e.g. $o(f(x)) \implies O(f(x))$),
\item linear orders on different instances of each kind of notation,
\item equivalence relations on asymptotically equivalent instances of each kind of notation,
\item equivalences of notations with their limit definitions,
\item \textbf{optional:} show properties of certain classes of instances - notably polynomials and exponentials,
\item add accompanying examples
\end{itemize}

\subsection{Limit definitions}

\subsection{Relations between the asymptotic sets}

\subsection{Linear orderings on asymptotic sets}

\section{The Master theorem}

Analyzing the time complexity of algorithms, especially recursive ones, is more often than not 
a non-trivial task. For a recursive algorithm, its time complexity can be written as a recurrence
formula, which is generally not easy, sometimes even impossible to solve with a closed formula.
In some cases, though, it turns out that we can find asymptotic bounds of the solution, despite not
being able to necessarily find the precise solution to the recurrence.

Divide-and-conquer algorithms, i.e. algorithms that work by recursively splitting the input problem
into similarly-sized subproblems have especially nice recurrence formulas which can be asymptotically
bounded with a simple formula that is known in the field as the Master theorem.

\begin{theorem}[Master theorem for divide-and-conquer recurrences]
TODO
\end{theorem}

Goals
\begin{itemize}
\item TODO
\end{itemize}
