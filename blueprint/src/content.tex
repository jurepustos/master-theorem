% In this file you should put the actual content of the blueprint.
% It will be used both by the web and the print version.
% It should *not* include the \begin{document}
%
% If you want to split the blueprint content into several files then
% the current file can be a simple sequence of \input. Otherwise It
% can start with a \section or \chapter for instance.

\chapter{Bachman-Landau notation and the Master theorem for divide-and-conquer recurrences}

The primary goal of this project is to formalize some results from computational complexity theory in Lean 4:
\begin{itemize}
\item the Bachman-Landau family of notations, e.g. big O and other closely related notations,
\item properties of the Bachman-Landau notations and relations between them,
\item the Master theorem for divide-and-conquer recurrences,
\item and, if time permits, the Akra-Bazzi method for solving a more general class of recurrences
\end{itemize}

\section{Bachman-Landau notation}

Bachman-Landau family of notations if the name of a few closely related notations used in 
algorithm analysis. The most famous of them is the so-called big-O notation. We define it as 
follows. Let $f: \mathbb{R} \to \mathbb{R}$. Then the set $O(f(x))$ contains a function 
$g: \mathbb{R} \to \mathbb{R}$ if there exist $x_0$ and positive $k$ such that 
$|g(x)| < k*|f(x)|$ for all $x \ge x_0$. We also say that $g(x) \in O(f(x))$ if $g(x)$ is 
asymptotically bounded above by $f(x)$. Similarly, we define $\Omega(f(x))$ as the set of functions 
$g(x)$ such that $|g(x)| \ge k*|f(x)|$ under the same assumptions, or that $g(x)$ is asymptotically 
bounded below by $f(x)$. If both $g(x) \in O(f(x))$ and $f(x) \in \Omega(f(x))$, we write 
$g(x) \in \Theta(f(x))$. By some abuse of notation, we sometimes write 
$g(x) = O(f(x))$, $g(x) = \Omega(f(x))$ and $g(x) = \Theta(f(x))$ respectively.

There are also some more notations that are less commonly used in algorithm analysis, but see some
use in other fields of mathematics. Out of these, we shall use two useful notations that fit 
the above notations elegantly. The first of these is the small-o notation - $g(x) \in o(f(x))$
if it is \textit{dominated} by $f(x)$. We can think of domination as a strict (i.e. exclusive) 
asymptotic bound. More precisely, $g(x) \in o(f(x))$ if for all positive $k$ there exists $x_0$ 
such that $|f(x)| < k*|g(x)|$ for all $x \ge x_0$. Mind the universal quantification over $k$ as
opposed to existential quantification in prior definitions. While $g(x) \in O(f(x))$ and implies
$\limsup_{x \to \infty} \frac{|f(x)|}{|g(x)|} < \infty$, $g(x) \in o(f(x))$ implies 
$lim_{x \to \infty} \frac{|f(x)|}{|g(x)|} = 0$. Analogously, we also have $g(x) \in \omega(f(x))$,
which states that $g(x)$ \textit{dominates} $f(x)$. The formal definition is just the definition
of $\Omega(f(x))$, but with an universal quantification on $k$. Precisely, $g(x) \in \omega(f(x))$
if for all positive $k$ there exists $x_0$ such that $|f(x)| \ge k*|g(x)|$ for all $x > x_0$.
And just like before, while $g(x) \in \Omega(f(x))$ implies 
$liminf_{x \to \infty} \frac{|f(x)|}{|g(x)|} > 0$, $g(x) \in \omega(f(x))$ implies 
$lim_{x \to \infty} \frac{|f(x)|}{|g(x)|} = \infty$.

Goals (to be updated as necessary)
\begin{itemize}
\item definitions of notations $O(f(x))$, $\Omega(f(x))$, $\Theta(f(x))$, $o(f(x))$ and $\omega(f(x))$,
\item basic relations between notations (e.g. $o(f(x)) \implies O(f(x))$),
\item linear orders on different instances of each kind of notation,
\item equivalence relations on asymptotically equivalent instances of each kind of notation,
\item equivalences of notations with their limit definitions
\item show properties of certain classes of instances - notably polynomials and exponentials
\item add accompanying examples
\end{itemize}


\section{The Master theorem}

Analyzing the time complexity of algorithms, especially recursive ones, is more often than not 
a non-trivial task. For a recursive algorithm, its time complexity can be written as a recurrence
formula, which is generally not easy, sometimes even impossible to solve with a closed formula.
In some cases, though, it turns out that we can find asymptotic bounds of the solution, despite not
being able to necessarily find the precise solution to the recurrence.

Divide-and-conquer algorithms, i.e. algorithms that work by recursively splitting the input problem
into similarly-sized subproblems have especially nice recurrence formulas which can be asymptotically
bounded with a simple formula that is known in the field as the Master theorem.

\begin{theorem}[Master theorem for divide-and-conquer recurrences]
TODO
\end{theorem}

Goals
\begin{itemize}
\item TODO
\end{itemize}
