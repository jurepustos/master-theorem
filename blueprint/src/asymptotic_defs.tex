\section{Basic asymptotic properties of functions}

\begin{definition}
    \label{def:asymp_pos}
    \lean{AsympPos}
    \leanok
    $f$ is asymptotically positive if there exists $x_0$ such that $f(x) > 0$
    for all $x \ge x_0$.

\end{definition}

\begin{definition}
    \label{def:asymp_neg}
    \lean{AsympNeg}
    \leanok
    $f$ is asymptotically negative if there exists $x_0$ such that $f(x) < 0$
    for all $x \ge x_0$.

\end{definition}

\begin{definition}
    \label{def:asymp_nonpos}
    \lean{AsympNonpos}
    \leanok
    $f$ is asymptotically nonpositive if there exists $x_0$ such that $f(x) \ge 0$
    for all $x \ge x_0$.

\end{definition}

\begin{definition}
    \label{def:asymp_nonneg}
    \lean{AsympNonneg}
    \leanok
    $f$ is asymptotically negative if there exists $x_0$ such that $f(x) \le 0$
    for all $x \ge x_0$.

\end{definition}

\begin{definition} 
    \label{def:asymp_le} 
    \lean{AsympLE}
    \leanok
    $f$ is asymptotically less than $g$ if there exists $x_0$ such that
    $f(x) \le g(x)$ for all $x \ge x_0$.

\end{definition}

\begin{definition}
    \label{def:asymp_ge}
    \lean{AsympGE}
    \leanok
    $f$ is asymptotically greater than $g$ if there exists $x_0$ such that
    $f(x) \ge g(x)$ for all $x \ge x_0$.

\end{definition}

\section{Asymptotic positivity and negativity}

\begin{lemma}
    \label{lemma:asymp_neg_of_pos}
    \lean{asymp_neg_of_pos}
    \leanok
    \uses{def:asymp_pos, def:asymp_neg}
    If $f$ is asymptotically negative, then $-f$ is asymptotically positive.
\end{lemma}

\begin{proof}
    \leanok
    By definition of asymptotic positivity, there exists an $x\_0$ such that $f(x) > 0$
    for all $x > x\_0$. It follows that $-f(x) > 0$, which is what is needed.
\end{proof}

\begin{lemma}
    \label{lemma:asymp_pos_of_neg}
    \lean{asymp_pos_of_neg}
    \leanok
    \uses{def:asymp_pos, def:asymp_neg}
    If $f$ is asymptotically positive, then $-f$ is asymptotically negative.
\end{lemma}

\begin{proof}
    \leanok
    By definition of asymptotic negativity, there exists an $x\_0$ such that $f(x) < 0$
    for all $x > x\_0$. It follows that $-f(x) < 0$, which is what is needed.
\end{proof}

\section{Asymptotic inequality}

\subsection{Positivity and negativity}

\begin{lemma}
    \label{lemma:asymp_pos_of_pos_le}
    \lean{asymp_pos_of_pos_le}
    \leanok
    \uses{def:asymp_pos, def:asymp_le}
    Let $f$ be asymptotically positive and let it be asymptotically less than $g$.
    Then $g$ is asymptotically positive.
\end{lemma}

\begin{proof}
    \leanok
    Assume with no loss of generality that $f(x) > 0$ for all $x > x\_0$ and that
    $f(y) \le g(y)$ for all $y > y\_0$. Let $z\_0 = \max \{ x\_0, y\_0 \}$. By transitivity
    of the inequality relations, we have $g(z) > 0$ for all $z > z\_0$. 
\end{proof}

\begin{lemma}
    \label{lemma:asymp_neg_of_le_neg}
    \lean{asymp_neg_of_le_neg}
    \leanok
    \uses{def:asymp_neg, def:asymp_le}
    Let $f$ be asymptotically less than $g$ and let $g$ be asymptotically negative.
    Then $f$ is asymptotically negative.
\end{lemma}

\begin{proof}
    \leanok
    Assume with no loss of generality that $g(x) < 0$ for all $x > x\_0$ and that
    $f(y) \le g(y)$ for all $y > y\_0$. Let $z\_0 = \max \{ x\_0, y\_0 \}$. By transitivity
    of the inequality relations, we have $f(z) < 0$ for all $z > z\_0$. 
\end{proof}

\begin{lemma}
    \label{lemma:asymp_pos_of_ge_pos}
    \lean{asymp_pos_of_ge_pos}
    \leanok
    \uses{def:asymp_pos, def:asymp_ge}
    Let $f$ be asymptotically greater than $g$ and let $g$ be asymptotically negative.
    Then $g$ is asymptotically negative.
\end{lemma}

\begin{proof}
    \leanok
    \uses{thm:asymp_le_iff_asymp_ge, lemma:asymp_pos_of_pos_le}
    By Lemma \ref{thm:asymp_le_iff_asymp_ge}, this statement is equivalent to Lemma
    \ref{lemma:asymp_pos_of_pos_le}.
\end{proof}

\begin{lemma}
    \label{lemma:asymp_neg_of_neg_ge}
    \lean{asymp_neg_of_neg_ge}
    \leanok
    \uses{def:asymp_neg, def:asymp_ge}
    Let $f$ be asymptotically negative and let it be asymptotically greater than $g$.
    Then $g$ is asymptotically negative.
\end{lemma}

\begin{proof}
    \leanok
    \uses{thm:asymp_le_iff_asymp_ge, lemma:asymp_neg_of_le_neg}
    By Theorem \ref{thm:asymp_le_iff_asymp_ge}, this statement is equivalent to Lemma
    \ref{lemma:asymp_neg_of_le_neg}.
\end{proof}


\subsection{Reflexivity}

\begin{lemma}
    \label{lemma:asymp_le_refl}
    \lean{asymp_le_refl}
    \leanok
    \uses{def:asymp_le}
    $f$ is asymptotically less than $f$.
\end{lemma}

\begin{proof}
    \leanok
    By reflexivity of $\le$, we have $f(x) \le f(x)$ for any given $x$.
\end{proof}

\begin{lemma}
    \label{lemma:asymp_ge_refl}
    \lean{asymp_ge_refl}
    \leanok
    \uses{def:asymp_ge}
    $f$ is asymptotically greater than $f$.
\end{lemma}

\begin{proof}
    \leanok
    By reflexivity of $\ge$, we have $f(x) \ge f(x)$ for any given $x$.
\end{proof}


\subsection{Equivalence}

\begin{lemma}
    \label{lemma:asymp_le_of_asymp_ge}
    \lean{asymp_le_of_asymp_ge}
    \leanok
    \uses{def:asymp_le, def:asymp_ge}
    Let $f$ be asymptotically less than $g$. Then $g$ is asymptotically
    greater than $f$.
\end{lemma}

\begin{proof}
    \leanok
    Since $f(x) \le g(x)$ for all $x > x\_0$, we have $g(x) \ge f(x)$.
\end{proof}

\begin{lemma}
    \label{lemma:asymp_ge_of_asymp_le}
    \lean{asymp_ge_of_asymp_le}
    \leanok
    \uses{def:asymp_le, def:asymp_ge}
    Let $f$ be asymptotically greater than $g$. Then $g$ is asymptotically
    less than $f$.
\end{lemma}

\begin{proof}
    \leanok
    Since $f(x) \ge g(x)$ for all $x > x\_0$, we have $g(x) \le f(x)$.
\end{proof}

\begin{theorem}
    \label{thm:asymp_le_iff_asymp_ge}
    \lean{asymp_le_iff_asymp_ge}
    \leanok
    \uses{def:asymp_le, def:asymp_ge}
    $f$ is asymptotically less than $g$ if and only if $g$ is asymptotically
    greater than $f$.
\end{theorem}

\begin{proof}
    \leanok
    \uses{lemma:asymp_le_of_asymp_ge, lemma:asymp_ge_of_asymp_le}
    Lemma \ref{lemma:asymp_le_of_asymp_ge} and Lemma \ref{lemma:asymp_ge_of_asymp_le} are both directions
    respectively.
\end{proof}


\subsection{Transitivity}

\begin{lemma}
    \label{lemma:asymp_le_trans}
    \lean{asymp_le_trans}
    \leanok
    If $f$ is asymptotically less than $g$ and $g$ is asymptotically less than $h$,
    then $f$ is asymptotically less than $h$.
\end{lemma}

\begin{proof}
    \leanok
    By assumption, $f(x) \le g(x)$ for all $x \ge x\_0$ and $g(y) \le h(y)$ for all
    $y \ge y\_0$. Let $z\_0 = \max \{ x\_0, y\_0 \}$. By transitivity, we have 
    $f(z) \le g(z)$ for all $z \ge z\_0$.
\end{proof}

\begin{lemma}
    \label{lemma:asymp_ge_trans}
    \lean{asymp_ge_trans}
    \leanok
    If $f$ is asymptotically greater than $g$ and $g$ is asymptotically greater than $h$,
    then $f$ is asymptotically greater than $h$.
\end{lemma}

\begin{proof}
    \leanok
    \uses{thm:asymp_le_iff_asymp_ge, lemma:asymp_le_trans}
    By the equivalence given by Lemma \ref{thm:asymp_le_iff_asymp_ge}, we can apply
    Lemma \ref{lemma:asymp_le_trans} in reverse, since $h$ is asymptotically less than
    $g$ and $g$ is asymptotically less than $f$.
\end{proof}


\subsection{Additivity}

\begin{lemma}
    \label{lemma:asymp_le_add}
    \lean{asymp_le_add}
    \leanok
    \uses{def:asymp_le}
    Let $f\_1$ be asymptotically less than $g\_1$ and $f\_2$ be asymptotically less
    than $g\_2$. Then $f\_1 + f\_2$ is asymptotically less than $g\_1 + g\_2$.
\end{lemma}

\begin{proof}
    \leanok
    Let $x\_0$ be such that $f\_1(x) \le g\_1(x)$ for all $x > x\_0$ and let $y\_0$ be
    such that $f\_2(y) \le g\_2(y)$ for all $y > y\_0$. Those exist due to assumptions.
    Now let $z\_0 = \max \{ x\_0, y\_0 \}$. By transitivity, $f\_1(z) \le g\_1(z)$ and 
    $f\_2(z) \le g\_2(z)$ for all $z > z\_0$. By additivity, we can merge both 
    inequalities by adding both terms on the left side and both terms on the right side.
    We thus get $f\_1(z) + f\_2(z) \le g\_1(z) + g\_2(z)$, which by definition and 
    and extensionality means that $f\_1 + f\_2$ is asymptotically less than $g\_1 + g\_2$.
\end{proof}

\begin{lemma}
    \label{lemma:asymp_ge_add}
    \lean{asymp_ge_add}
    \leanok
    \uses{def:asymp_ge}
    Let $f\_1$ be asymptotically greater than $g\_1$ and $f\_2$ be asymptotically greater
    than $g\_2$. Then $f\_1 + f\_2$ is asymptotically greater than $g\_1 + g\_2$.
\end{lemma}

\begin{proof}
    \leanok
    \uses{thm:asymp_le_iff_asymp_ge, def:asymp_le, lemma:asymp_le_add}
    By Theorem \ref{thm:asymp_le_iff_asymp_ge}, $g\_1$ and $g\_2$ are asymptotically less 
    than $f\_1$ and $f\_2$ respectively. It suffices to show that $g\_1 + g\_2$ is
    asymptotically less than $f\_1 + f\_2$, which is precisely the result of
    Lemma \ref{lemma:asymp_le_add}.
\end{proof}

\begin{lemma}
    \label{lemma:asymp_ge_add_pos}
    \lean{asymp_ge_add_pos}
    \leanok
    \uses{def:asymp_pos, def:asymp_ge}
    Let $f\_1$ be asymptotically positive. Let also $f\_2$ be asymptotically greater 
    than $g$. Then $f\_1 + f\_2$ is asymptotically greater than $g$.
\end{lemma}

\begin{proof}
    \leanok
    By definition, there exists some $x\_0$ such that $f\_1(x) > 0$ for all $x > x\_0$.
    We also have $f\_2(y) \ge g(y)$ for all $y > y\_0$ for some $y\_0$. Let
    $z\_0 = \max \{ x\_0, y\_0, \}$. We now have, for all $z > z\_0$ both $f\_1(z) > 0$
    and $f\_2(z) \ge g(z)$. By additivity, we have $f\_1(z) + f\_2(z) \ge g(z)$. 
\end{proof}

\begin{lemma}
    \label{lemma:asymp_le_add_neg}
    \lean{asymp_le_add_neg}
    \leanok
    \uses{def:asymp_neg, def:asymp_ge}
    Let $f\_1$ be asymptotically negative. Let also $f\_2$ be asymptotically less 
    than $g$. Then $f\_1 + f\_2$ is asymptotically less than $g$.
\end{lemma}

\begin{proof}
    \leanok
    By definition, there exists some $x\_0$ such that $f\_1(x) < 0$ for all $x > x\_0$.
    We also have $f\_2(y) \le g(y)$ for all $y > y\_0$ for some $y\_0$. Let
    $z\_0 = \max \{ x\_0, y\_0, \}$. We now have, for all $z > z\_0$ both $f\_1(z) < 0$
    and $f\_2(z) \le g(z)$. By additivity, we have $f\_1(z) + f\_2(z) \le g(z)$. 
\end{proof}


\subsection{Scalar multiplicativity}

\begin{lemma}
    \label{lemma:asymp_le_pos_smul}
    \lean{asymp_le_pos_smul}
    \leanok
    \uses{def:asymp_le}
    Let $c > 0$ and let $f$ be asymptotically less than $g$. Then $c \cdot f$ is asymptotically
    less than $c \cdot g$.
\end{lemma}

\begin{proof}
    \leanok
    This is a simple consequence of scalar multiplication by a positive constant.
\end{proof}

\begin{lemma}
    \label{lemma:asymp_ge_pos_smul}
    \lean{asymp_ge_pos_smul}
    \leanok
    \uses{def:asymp_ge}
    Let $c > 0$ and let $f$ be asymptotically greater than $g$. Then $c \cdot f$ is asymptotically
    greater than $c \cdot g$.
\end{lemma}

\begin{proof}
    \leanok
    \uses{thm:asymp_le_iff_asymp_ge, lemma:asymp_le_pos_smul}
    By applying Theorem \ref{thm:asymp_le_iff_asymp_ge}, the proof boils down to proving that
    $c \cdot g$ is asymptotically less than $c \cdot f$, which is precisely shown
    by Lemma \ref{lemma:asymp_le_pos_smul}.
\end{proof}

\begin{lemma}
    \label{lemma:asymp_le_neg_smul}
    \lean{asymp_le_neg_smul}
    \leanok
    \uses{def:asymp_neg, def:asymp_le, def:asymp_ge}
    Let $c < 0$ and let $f$ be asymptotically less than $g$. Then $c \cdot f$ is asymptotically
    greater than $c \cdot g$.
\end{lemma}

\begin{proof}
    \leanok
    This is a simple consequence of the fact that if $f(x) \le g(x)$, then for a $c < 0$
    we have $c \cdot f(x) \ge c \cdot g(x)$.
\end{proof}

\begin{lemma}
    \label{lemma:asymp_ge_neg_smul}
    \lean{asymp_ge_neg_smul}
    \leanok
    \uses{def:asymp_neg, def:asymp_ge, def:asymp_le}
    Let $c < 0$ and let $f$ be asymptotically greater than $g$. Then $c \cdot f$ is asymptotically
    less than $c \cdot g$.
\end{lemma}

\begin{proof}
    \leanok
    \uses{thm:asymp_le_iff_asymp_ge, lemma:asymp_le_neg_smul}
    Similar to above, the proof is a direct application of Theorem \ref{thm:asymp_le_iff_asymp_ge}
    and Lemma \ref{lemma:asymp_le_neg_smul}.
\end{proof}


\subsection{Multiplicativity}

\begin{theorem}
    \label{thm:asymp_le_nonneg_mul}
    \lean{asymp_le_nonneg_mul}
    \leanok
    \uses{def:asymp_nonneg, def:asymp_le}
    Let $f\_1$ and $f\_2$ be asymptotically nonnegative functions. If f$\_1$ is asymptotically
    less than $g\_1$ and $f\_2$ is asymptotically less than $g\_2$, then $f\_1 * f\_2$
    is asymptotically less than $g\_1 * g\_2$.
\end{theorem}

\begin{proof}
    \leanok
    Asymptotic properties give constants $x\_i, 0 \le i \le 3$, above which each property holds.
    We take $x_M = \max_{0 \le i \le 3} x_i$ as the constant of the needed property. 
    For all $x > x_M$, all given asymptotic properties hold, so the wanted property holds
    by properties of the inequality relation.
\end{proof}

\begin{theorem}
    \label{thm:asymp_ge_nonpos_mul}
    \lean{asymp_ge_nonpos_mul}
    \leanok
    \uses{def:asymp_nonpos, def:asymp_ge, def:asymp_le}
    Let $f\_1$ and $f\_2$ be asymptotically nonpositive functions. If f$\_1$ is asymptotically
    greater than $g\_1$ and $f\_2$ is asymptotically greater than $g\_2$, then $f\_1 * f\_2$
    is asymptotically less than $g\_1 * g\_2$.
\end{theorem}

\begin{proof}
    \leanok
    Analogously to above, the proof comes from taking the maximum of asymptotic constants 
    as the asymptotic lower bound for nonpositivity. This time, however, the inequality
    flips due to nonpositive terms in $f\_1(n) * f\_2(n) \le g\_1(n) * g\_2(n)$, since 
    $n$ is larger than all of the asymptotic lower bounds.
\end{proof}
