section{Asymptotic growth}
\begin{definition}
    \label{def:asymp_bounded_above}
    \lean{AsympBoundedAbove}
    \leanok
    \uses{def:asymp_le}
    $f$ is asymptotically bounded above by $g$ if there exists a $k > 0$ 
    such that $f$ is asymptotically less than $k*g$.
\end{definition}

\begin{definition}
    \label{def:asymp_bounded_below}
    \lean{AsympBoundedBelow}
    \leanok
    \uses{def:asymp_ge}
    $f$ is asymptotically bounded below by $g$ if there exists $k > 0$ 
    such that $f$ is asymptotically greater than $k*g$.
\end{definition}

\begin{definition}
    \label{def:asymp_bounded}
    \lean{AsympBounded}
    \leanok
    \uses{def:asymp_bounded_above, def:asymp_bounded_below}
    $f$ is asymptotically bounded by $g$ if $f$ is asymptotically bounded
    above and below by $g$.
\end{definition}

\begin{definition}
    \label{def:asymp_right_dom}
    \lean{AsympRightDom}
    \leanok
    \uses{def:asymp_le}
    $f$ is asymptotically dominated by $g$ if for all $k > 0$ $f$ is asymptotically 
    less than $k*g$.

\end{definition}

\begin{definition}
    \label{def:asymp_left_dom}
    \lean{AsympLeftDom}
    \leanok
    \uses{def:asymp_ge}
    $f$ asymptotically dominates $g$ if for all $k > 0$ $(x)$ is asymptotically
    greater than $k*g$.

\end{definition}

\begin{lemma}
    \label{lemma:asymp_bounded_above_of_right_dom}
    \lean{asymp_bounded_above_of_right_dom}
    \leanok
    \uses{def:asymp_right_dom, def:asymp_bounded_above}
    If $f$ is dominated by $g$, then it's also bounded above by $g$.
\end{lemma}

\begin{proof}
    \leanok 
    The definitions of $f$ being dominated and bounded above by $g$ only differ
    in the quantifier before $k$ at the very start (universal for the hypothesis, existential
    for the goal), so it suffices to use any positive value for $k$. We can use $1$. 
    The desired result then follows directly.
\end{proof}

\begin{lemma}
    \label{lemma:asymp_bounded_below_of_left_dom}
    \lean{asymp_bounded_below_of_left_dom}
    \leanok
    \uses{def:asymp_left_dom, def:asymp_bounded_below}
    If $f$ dominates $g$, then it's bounded below by $g$.
\end{lemma}

\begin{proof}
    \leanok
    The proof is entirely analogous to the previous proof.
\end{proof}

\begin{lemma}
    \label{lemma:asymp_bounded_iff}
    \lean{asymp_bounded_iff}
    \leanok
    \uses{def:asymp_bounded_above, def:asymp_bounded_below, def:asymp_bounded}
    $f$ is asymptotically bounded above and below by $g$ if and only if $f$ is 
    asymptotically bounded by $g$.
\end{lemma}

\begin{proof}
    \leanok
    Both directions follow directly from the definition of asymptotic boundedness. 
\end{proof}

\begin{lemma}
    \label{lemma:not_asymp_pos_bounded_below_and_right_dom}
    \lean{not_asymp_pos_bounded_below_and_right_dom}
    \leanok
    \uses{def:asymp_pos, def:asymp_bounded_below, def:asymp_right_dom}
    Let $g$ be asymptotically positive. Then $f$ is not both asymptotically bounded 
    below by $g$ and asymptotically dominated by $g$.
\end{lemma}

\begin{proof}
    \leanok
    Suppose $f$ is asymptotically bounded below by $g$ and also asymptotically dominated
    by $g$. We need to find a contradiction to prove the statement. 

    First, we claim that there exists $x\_0$ such that for all $x \ge x\_0$ we have 
    $f(x) \ge k \cdot g(x)$ for some $k > 0$, $g(x) \ge 0$ and $f(x) \le k' \cdot g(x)$ 
    for all $k' > 0$. Each of the asymptotic assumption gives one such constant, so
    taking the maximum of all three gives the needed value.

    For the contradiction, consider $f(x) \le k' \cdot g(x)$ when $k' = k / 2$.
    In this case, we have $f(x) \le (k / 2) \cdot g(x)$, but we also have
    $k \cdot g(x) \le f(x)$, leading by transitivity to $k \cdot g(x) \le (k / 2) \cdot g(x)$,
    an obvious contradiction to the fact that $(k / 2) \cdot g(x) \le k \cdot g(x)$.
\end{proof}

\begin{theorem}
    \label{thm:not_asymp_pos_right_dom_of_bounded_below}
    \lean{not_asymp_pos_right_dom_of_bounded_below}
    \leanok
    \uses{def:asymp_pos, def:asymp_bounded_above, def:asymp_right_dom}
    Let $f$ be asymptotically positive. If $f$ is asymptotically bounded below by $g$, 
    then $f$ is not asymptotically dominated by $g$.
\end{theorem}

\begin{proof}
    \leanok
    \uses{lemma:not_asymp_pos_bounded_below_and_right_dom}
    This is a direct application of Lemma \ref{lemma:not_asymp_pos_bounded_below_and_right_dom}. 
\end{proof}

\begin{theorem}
    \label{thm:not_asymp_pos_bounded_below_of_right_dom}
    \lean{not_asymp_pos_bounded_below_of_right_dom}
    \leanok
    \uses{def:asymp_pos, def:asymp_bounded_above, def:asymp_right_dom}
    Let $f$ be asymptotically positive. If $f$ is asymptotically bounded below by $g$, 
    then $f$ is not asymptotically dominated by $g$.
\end{theorem}

\begin{proof}
    \leanok
    \uses{lemma:not_asymp_pos_bounded_below_and_right_dom}
    This is a direct application of Lemma \ref{lemma:not_asymp_pos_bounded_below_and_right_dom}. 
\end{proof}

\begin{lemma}
    \label{lemma:not_asymp_pos_bounded_above_and_left_dom}
    \lean{not_asymp_pos_bounded_above_and_left_dom}
    \leanok
    \uses{def:asymp_pos, def:asymp_bounded_above, def:asymp_left_dom}
    Let $g$ be asymptotically positive. Then it does not hold that both $f$ is asymptotically 
    bounded above by $g$ and $f$ asymptotically dominate $g$.
\end{lemma}

\begin{proof}
    \leanok
    The proof is analogous to the proof of Lemma 
    \ref{lemma:not_asymp_pos_bounded_below_and_right_dom}. This time, we set $k' = k + 1$
    and thus produce the false inequality $(k + 1) \cdot g(x) \le k \cdot g(x)$.
\end{proof}

\begin{theorem}
    \label{thm:not_asymp_left_dom_of_bounded_above_pos}
    \lean{not_asymp_left_dom_of_bounded_above_pos}
    \leanok
    \uses{def:asymp_pos, def:asymp_bounded_below, def:asymp_left_dom}
    Let $f$ be asymptotically positive. If $f$ is bounded below by $g$, then $f$ 
    does not dominate $g$.
\end{theorem}

\begin{proof}
    \leanok
    \uses{lemma:not_asymp_pos_bounded_above_and_left_dom}
    This is a direct application of Lemma \ref{lemma:not_asymp_pos_bounded_above_and_left_dom}. 
\end{proof}

\begin{theorem}
    \label{thm:not_asymp_bounded_above_of_left_dom_pos}
    \lean{not_asymp_bounded_above_of_left_dom_pos}
    \leanok
    \uses{def:asymp_pos, def:asymp_bounded_below, def:asymp_left_dom}
    Let $f$ be asymptotically positive. If $f$ is asymptotically bounded above by $g$, 
    then $f$ does not asymptotically dominate $g$.
\end{theorem}

\begin{proof}
    \leanok
    \uses{lemma:not_asymp_pos_bounded_above_and_left_dom}
    This is a direct application of Lemma \ref{lemma:not_asymp_pos_bounded_above_and_left_dom}. 
\end{proof}

\begin{lemma}
    \label{lemma:not_asymp_pos_left_and_right_dom}
    \lean{not_asymp_pos_left_and_right_dom}
    \leanok
    \uses{def:asymp_pos, def:asymp_left_dom, def:asymp_right_dom}
    Let $g$ be asymptotically positive. Then it is not true that both $f$ is asymptotically
    dominated by $g$ and that $f$ dominates $g$.
\end{lemma}

\begin{proof}
    Suppose $f$ both dominates $g$ and is dominated by $g$. Our goal is to find a
    contradiction. We have by definition the inequalities $g(x) > 0$, 
    $f(x) \ge k\_1 \cdot g(x)$ and $f(x) \le k\_2 \cdot g(x)$ for all $k\_1 > 0$, 
    $k\_2 > 0$ and for all $x \ge x\_0$ for some $x\_0$. Note that we use the same 
    constant $x\_0$ for all inequalities with no loss of generality. In fact, we shall also use the same $x\_0$ 
    in the asymptotic positivity condition $g(x) > 0$.

    Fix $k\_1 = 2$ and $k\_2 = 1$ (generally, we only need $k\_1 \ge k\_2$), so we now have
    $f(x) \ge 2 \cdot g(x)$ and $f(x) \le g(x)$. From these inequalities, it immediately
    follows that $2 \cdot g(x) \le g(x)$. However, since $1 \le 2$ and $g(x) > 0$, we have
    $g(x) < 2 \cdot g(x)$. We thus have two contradicting inequalities, finishing the proof.
\end{proof}

\begin{theorem}
    \label{thm:not_asymp_pos_left_dom_of_right_dom}
    \lean{not_asymp_pos_left_dom_of_right_dom}
    \leanok
    \uses{def:asymp_pos, def:asymp_left_dom, def:asymp_right_dom}
    Let $g$ be asymptotically positive. If $f$ is asymptotically dominated by $g$, then
    $f$ does not asymptotically dominate $g$.
\end{theorem}

\begin{proof}
    \leanok
    \uses{lemma:not_asymp_pos_left_and_right_dom}
    This is a direct application of Lemma \ref{lemma:not_asymp_pos_left_and_right_dom}. 
\end{proof}

\begin{theorem}
    \label{thm:not_asymp_pos_right_dom_of_left_dom}
    \lean{not_asymp_pos_right_dom_of_left_dom}
    \leanok
    \uses{def:asymp_pos, def:asymp_left_dom, def:asymp_right_dom}
    Let $g$ be asymptotically positive. If $f$ asymptotically dominates $g$, then
    $f$ is not asymptotically dominated by $g$.
\end{theorem}

\begin{proof}
    \leanok
    \uses{lemma:not_asymp_pos_left_and_right_dom}
    This is a direct application of Lemma \ref{lemma:not_asymp_pos_left_and_right_dom}. 
\end{proof}


\subsection{Reflexivity}

\begin{lemma}
    \label{lemma:asymp_bounded_refl}
    \lean{asymp_bounded_refl}
    \leanok
    \uses{def:asymp_bounded}
    $f$ is asymptotically bounded by itself. 

\end{lemma}

\begin{proof}
    \leanok
    \uses{def:asymp_bounded, def:asymp_bounded_above, def:asymp_bounded_below}
    Proving asymptotic boundedness is equivalent to proving boundedness above and below.
    Both can be proved the same way - we choose $1_K$ for $K$ and $1_R$ for $N$, then the
    required asymptotic growth properties follow from definitions of identity elements for 
    $K$ and $R$.
\end{proof}

\begin{lemma}
    \label{lemma:asymp_bounded_above_refl}
    \lean{asymp_bounded_above_refl}
    \leanok
    \uses{def:asymp_bounded_above}
    $f$ is asymptotically bounded above by itself.
\end{lemma}

\begin{proof}
    \leanok
    \uses{lemma:asymp_bounded_refl}
    This follows directly from \ref{lemma:asymp_bounded_refl}.
\end{proof}

\begin{lemma}
    \label{lemma:asymp_bounded_below_refl}
    \lean{asymp_bounded_below_refl}
    \leanok
    \uses{def:asymp_bounded_below}
    $f$ is asymptotically bounded below by itself.
\end{lemma}

\begin{proof}
    \leanok
    \uses{lemma:asymp_bounded_refl}
    This follows directly from \ref{lemma:asymp_bounded_refl}.
\end{proof}


\subsection{Transitivity}

\begin{lemma}
    \label{lemma:asymp_bounded_above_trans}
    \lean{asymp_bounded_above_trans}
    \leanok
    \uses{def:asymp_bounded_above}
    If $f$ is asymptotically bounded above by $g$ and $g$ is asymptotically bounded above
    by $h$, then $f$ is asymptotically bounded above by $h$.
\end{lemma}

\begin{proof}
    \leanok
    \uses{lemma:asymp_le_pos_smul, lemma:asymp_le_trans}
    Let $k\_1$ and $k\_2$ be the constants such that $f(x) \le k\_1 \cdot g(x)$ and
    $g(x) \le k\_2 \cdot h(x)$ for sufficiently large $x$. By multiplicativity and 
    transitivity, we have $f(x) \le k\_1 \cdot k\_2 \cdot h(x)$.
\end{proof}

\begin{lemma}
    \label{lemma:asymp_bounded_below_trans}
    \lean{asymp_bounded_below_trans}
    \leanok
    \uses{def:asymp_bounded_below}
    If $f$ is asymptotically bounded below by $g$ and $g$ is asymptotically bounded below
    by $h$, then $f$ is asymptotically bounded below by $h$.
\end{lemma}

\begin{proof}
    \leanok
    \uses{lemma:asymp_ge_pos_smul, lemma:asymp_ge_trans}
    Let $k\_1$ and $k\_2$ be the constants such that $f(x) \ge k\_1 \cdot g(x)$ and
    $g(x) \ge k\_2 \cdot h(x)$ for sufficiently large $x$. By multiplicativity and 
    transitivity, we have $f(x) \ge k\_1 \cdot k\_2 \cdot h(x)$.
\end{proof}

\begin{lemma}
    \label{lemma:asymp_bounded_trans}
    \lean{asymp_bounded_trans}
    \leanok
    \uses{def:asymp_bounded}
    If $f$ is asymptotically bounded by $g$ and $g$ is asymptotically bounded 
    by $h$, then $f$ is asymptotically bounded by $h$.
\end{lemma}

\begin{proof}
    \leanok
    \uses{lemma:asymp_bounded_above_trans, lemma:asymp_bounded_below_trans}
    A direct consequence of Lemma \ref{lemma:asymp_bounded_above_trans} and Lemma
    \ref{lemma:asymp_bounded_below_trans}.
\end{proof}

\begin{lemma}
    \label{lemma:asymp_right_dom_trans}
    \lean{asymp_right_dom_trans}
    \leanok
    \uses{def:asymp_right_dom}
    If $f$ is asymptotically dominated by $g$ and $g$ is asymptotically dominated by $h$, then 
    $f$ is asymptotically dominated by $h$.
\end{lemma}

\begin{proof}
    \leanok
    \uses{lemma:asymp_le_pos_smul, lemma:asymp_le_trans}
    Let $k > 0$. Then by first assumption, we have $f(x) \le k \cdot g(x)$ for large 
    $x$. By the second assumption, we have $g(x) \le h(x)$ for large $x$. By multiplicativity 
    and transitivity, we get $f(x) \le k \cdot h(x)$.
\end{proof}

\begin{lemma}
    \label{lemma:asymp_left_dom_trans}
    \lean{asymp_left_dom_trans}
    \leanok
    \uses{def:asymp_left_dom}
    If $f$ asymptotically dominates $g$ and $g$ asymptotically dominates $h$, then 
    $f$ asymptotically dominates $h$.
\end{lemma}

\begin{proof}
    \leanok
    \uses{lemma:asymp_ge_pos_smul, lemma:asymp_ge_trans}
    Let $k > 0$. Then by first assumption, we have $f(x) \ge k \cdot g(x)$ for large 
    $x$. By the second assumption, we have $g(x) \ge h(x)$ for large $x$. By multiplicativity 
    and transitivity, we get $f(x) \ge k \cdot h(x)$.
\end{proof}


\subsection{Scalar multiplicativity}

\begin{lemma}
    \label{lemma:asymp_bounded_above_pos_smul}
    \lean{asymp_bounded_above_pos_smul}
    \leanok
    \uses{def:asymp_bounded_above}
    Let $c > 0$. If $f$ is bounded above by $g$, then $c \cdot f$ is also bounded
    above by $g$.
\end{lemma}

\begin{proof}
    \leanok
    \uses{lemma:asymp_le_pos_smul}
    Let $k$ be the constant such that $f$ is asymptotically less than $k \cdot g$.
    By positive scalar multiplicativity of asymptotic inequality, $c \cdot f$ is 
    asymptotically less than $c \cdot k \cdot g$. Since $c \cdot k > 0$, this implies 
    that $c \cdot f$ is bounded above by $g$.
\end{proof}

\begin{lemma}
    \label{lemma:asymp_bounded_below_pos_smul}
    \lean{asymp_bounded_below_pos_smul}
    \leanok
    \uses{def:asymp_bounded_below}
    Let $c > 0$. If $f$ is bounded below by $g$, then $c \cdot f$ is also bounded
    below by $g$.
\end{lemma}

\begin{proof}
    \leanok
    \uses{lemma:asymp_ge_pos_smul}
    Let $k$ be the constant such that $f$ is asymptotically greater than $k \cdot g$.
    By positive scalar multiplicativity of asymptotic inequality, $c \cdot f$ is 
    asymptotically greater than $c \cdot k \cdot g$. Since $c \cdot k > 0$, this implies
    that $c \cdot f$ is bounded below by $g$.
\end{proof}

\begin{lemma}
    \label{lemma:asymp_bounded_pos_smul}
    \lean{asymp_bounded_pos_smul}
    \leanok
    \uses{def:asymp_bounded}
    Let $c > 0$. If $f$ is bounded by $g$, then $c \cdot f$ is also bounded by $g$.
\end{lemma}

\begin{proof}
    \leanok
    \uses{lemma:asymp_bounded_above_pos_smul, lemma:asymp_bounded_below_pos_smul,}
    Above boundedness is exactly Lemma \ref{lemma:asymp_bounded_above_pos_smul} and
    below boundedness is exactly Lemma \ref{lemma:asymp_bounded_below_pos_smul}.
\end{proof}

\begin{lemma}
    \label{lemma:asymp_bounded_above_neg_smul}
    \lean{asymp_bounded_above_neg_smul}
    \leanok
    \uses{def:asymp_bounded_above}
    Let $c < 0$. If $f$ is bounded above by $g$, then $c \cdot f$ is bounded
    above by $-g$.
\end{lemma}

\begin{proof}
    \leanok
    \uses{lemma:asymp_le_neg_smul}
    Let $k$ be the constant such that $f$ is asymptotically less than $k \cdot g$.
    By positive scalar multiplicativity of asymptotic inequality, $-c \cdot f$ is 
    asymptotically less than $-c \cdot k \cdot g$. Since $-c \cdot k > 0$, this implies
    that $c \cdot f$ is bounded above by $-g$.
\end{proof}

\begin{lemma}
    \label{lemma:asymp_bounded_below_neg_smul}
    \lean{asymp_bounded_below_neg_smul}
    \leanok
    \uses{def:asymp_bounded_below}
    Let $c < 0$. If $f$ is bounded below by $g$, then $c \cdot f$ is bounded
    below by $-g$.
\end{lemma}

\begin{proof}
    \leanok
    \uses{lemma:asymp_le_neg_smul}
    Let $k$ be the constant such that $f$ is asymptotically greater than $k \cdot g$.
    By positive scalar multiplicativity of asymptotic inequality, $-c \cdot f$ is 
    asymptotically greater than $-c \cdot k \cdot g$. Since $-c \cdot k > 0$, this implies
    that $c \cdot f$ is bounded below by $-g$.
\end{proof}

\begin{lemma}
    \label{lemma:asymp_bounded_neg_smul}
    \lean{asymp_bounded_neg_smul}
    \leanok
    \uses{def:asymp_bounded}
    Let $c < 0$. If $f$ is bounded by $g$, then $c \cdot f$ is bounded by $-g$.
\end{lemma}

\begin{proof}
    \leanok
    \uses{lemma:asymp_bounded_above_neg_smul, lemma:asymp_bounded_below_neg_smul}
    Above boundedness is exactly Lemma \ref{lemma:asymp_bounded_above_neg_smul} and
    below boundedness is exactly Lemma \ref{lemma:asymp_bounded_below_neg_smul}.
\end{proof}


\subsection{Additivity}

\begin{lemma}
    \label{lemma:asymp_bounded_above_add}
    \lean{asymp_bounded_above_add}
    \leanok
    \uses{def:asymp_bounded_above}
    Let $f\_1$ and $f\_2$ be bounded above by $g$. Then $f\_1 + f\_2$ is also bounded 
    above by $g$.
\end{lemma}

\begin{proof}
    \leanok
    \uses{lemma:asymp_le_add}
    Let $k\_1$ and $k\_2$ be the constants such that $f\_1$ is asymptotically less than $k\_1 \cdot g$ 
    and $f\_2$ is asymptotically less than $k\_2 \cdot g$. By additivity of asymptotic
    inequality, $f\_1 + f\_2$ is asymptotically less than $(k\_1 * k\_2) \cdot g$.
    It directly follows that $f\_1 + f\_2$ is asymptotically bounded above by $g$.
\end{proof}

\begin{lemma}
    \label{lemma:asymp_bounded_below_add}
    \lean{asymp_bounded_below_add}
    \leanok
    \uses{def:asymp_bounded_below}
    Let $f\_1$ and $f\_2$ be bounded below by $g$. Then $f\_1 + f\_2$ is also bounded 
    below by $g$.
\end{lemma}

\begin{proof}
    \leanok
    \uses{lemma:asymp_ge_add}
    Let $k\_1$ and $k\_2$ be the constants such that $f\_1$ is asymptotically greater than $k\_1 \cdot g$ 
    and $f\_2$ is asymptotically greater than $k\_2 \cdot g$. By additivity of asymptotic
    inequality, $f\_1 + f\_2$ is asymptotically greater than $(k\_1 * k\_2) \cdot g$.
    It directly follows that $f\_1 + f\_2$ is asymptotically bounded below by $g$.
\end{proof}

\begin{lemma}
    \label{lemma:asymp_bounded_add}
    \lean{asymp_bounded_above_add}
    \leanok
    \uses{def:asymp_bounded}
    Let $f\_1$ and $f\_2$ be bounded by $g$. Then $f\_1 + f\_2$ is also bounded by $g$.
\end{lemma}

\begin{proof}
    \leanok
    \uses{lemma:asymp_bounded_above_add, lemma:asymp_bounded_below_add}
    This is proved directly by Lemma \ref{lemma:asymp_bounded_above_add} and Lemma
    \ref{lemma:asymp_bounded_below_add}.
\end{proof}

\begin{lemma}
    \label{lemma:asymp_bounded_below_add_pos}
    \lean{asymp_bounded_below_add_pos}
    \leanok
    \uses{def:asymp_bounded_below, def:asymp_pos}
    Let $f\_1$ be bounded below by $g$ and let $f\_2$ be asymptotically positive. 
    Then $f\_1 + f\_2$ is bounded below by $g$.
\end{lemma}

\begin{proof}
    \leanok
    \uses{lemma:asymp_ge_add_pos}
    This property immediately follows from Lemma \ref{lemma:asymp_ge_add_pos}.
\end{proof}

\begin{lemma}
    \label{lemma:asymp_bounded_above_add_neg}
    \lean{asymp_bounded_above_add_neg}
    \leanok
    \uses{def:asymp_bounded_above, def:asymp_neg}
    Let $f\_1$ be bounded above by $g$ and let $f\_2$ be asymptotically negative. 
    Then $f\_1 + f\_2$ is bounded above by $g$.
\end{lemma}

\begin{proof}
    \leanok
    \uses{lemma:asymp_le_add_neg}
    This property immediately follows from Lemma \ref{lemma:asymp_le_add_neg}.
\end{proof}

\begin{lemma}
    \label{lemma:asymp_bounded_add_pos_above}
    \lean{asymp_bounded_add_pos_above}
    \leanok
    \uses{def:asymp_bounded, def:asymp_bounded_above, def:asymp_pos}
    Let $f\_1$ be bounded by $g$. Let also $f\_2$ be asymptotically positive and
    bounded above by $g$. Then $f\_1 + f\_2$ is bounded by $g$.
\end{lemma}

\begin{proof}
    \leanok
    \uses{lemma:asymp_bounded_above_add, lemma:asymp_bounded_below_add_pos}
    This property immediately follows from Lemma \ref{lemma:asymp_bounded_above_add}
    and Lemma \ref{lemma:asymp_bounded_below_add_pos}.
\end{proof}

\begin{lemma}
    \label{lemma:asymp_bounded_add_neg_below}
    \lean{asymp_bounded_add_neg_below}
    \leanok
    \uses{def:asymp_bounded, def:asymp_bounded_below, def:asymp_neg}
    Let $f\_1$ be bounded by $g$. Let also $f\_2$ be asymptotically negative and
    bounded below by $g$. Then $f\_1 + f\_2$ is bounded by $g$.
\end{lemma}

\begin{proof}
    \leanok
    \uses{lemma:asymp_bounded_below_add, lemma:asymp_bounded_above_add_neg}
    This property immediately follows from Lemma \ref{lemma:asymp_bounded_below_add}
    and Lemma \ref{lemma:asymp_bounded_above_add_neg}.
\end{proof}

\begin{lemma}
    \label{lemma:asymp_bounded_add_pos_right_dom}
    \lean{asymp_bounded_add_pos_above}
    \leanok
    \uses{def:asymp_bounded, def:asymp_right_dom, def:asymp_pos}
    Let $f\_1$ be bounded by $g$. Let also $f\_2$ be asymptotically positive and
    let $f\_2$ be dominated by $g$. Then $f\_1 + f\_2$ is bounded by $g$.
\end{lemma}

\begin{proof}
    \leanok
    \uses{thm:asymp_bounded_add_pos_above, lemma:asymp_bounded_above_of_right_dom}
    We prove this with an application of Lemma \ref{lemma:asymp_bounded_add_pos_above}
    on Lemma \ref{lemma:asymp_bounded_above_of_right_dom}.
\end{proof}

\begin{lemma}
    \label{lemma:asymp_bounded_add_neg_left_dom}
    \lean{asymp_bounded_add_neg_left_dom}
    \leanok
    \uses{def:asymp_bounded, def:asymp_bounded_below, def:asymp_neg}
    Let $f\_1$ be bounded by $g$. Let also $f\_2$ be asymptotically negative and
    let $f\_2$ dominate $g$. Then $f\_1 + f\_2$ is bounded by $g$.
\end{lemma}

\begin{proof}
    \leanok
    \uses{thm:asymp_bounded_add_neg_below, lemma:asymp_bounded_below_of_left_dom}
    We prove this with an application of Lemma \ref{lemma:asymp_bounded_add_neg_below}
    on Lemma \ref{lemma:asymp_bounded_below_of_left_dom}.
\end{proof}


\subsection{Multiplicativity}

\begin{lemma}
    \label{lemma:asymp_bounded_above_nonneg_mul}
    \lean{asymp_bounded_above_nonneg_mul}
    \leanok
    \uses{def:asymp_bounded_above, def:asymp_nonneg}
    Let $f\_1$ and $f\_2$ be asymptotically nonnegative functions such that $f\_1$ 
    is asymptotically bounded above by $g\_1$ and $f\_2$ is asymptotically bounded 
    above by $g\_2$. Then $f\_1 * f\_2$ is asymptotically bounded above by $g\_1 * g\_2$.
\end{lemma}

\begin{proof}
    \leanok
    \uses{lemma:asymp_le_nonneg_mul}
    Let $k\_1$ and $k\_2$ be the constants such that $f\_1$ is asymptotically less than
    $g\_1$ and $f\_2$ is asymptotically less than $g\_2$. By multiplicativity of asymptotic 
    inequality, $f\_1 * f\_2$ is asymptotically less than $k\_1 \cdot g\_1 * k\_2 \cdot g\_2$,
    which is equivalent to asymptotic above boundedness of $f\_1 * f\_2$ by $g\_1 * g\_2$.
\end{proof}

\begin{lemma}
    \label{lemma:asymp_bounded_below_nonpos_mul}
    \lean{asymp_bounded_below_nonpos_mul}
    \leanok
    \uses{def:asymp_bounded_below, def:asymp_nonpos}
    Let $f\_1$ and $f\_2$ be asymptotically nonpositive functions such that $f\_1$ 
    is asymptotically bounded below by $g\_1$ and $f\_2$ is asymptotically bounded 
    below by $g\_2$. Then $f\_1 * f\_2$ is asymptotically bounded below by $g\_1 * g\_2$.
\end{lemma}

\begin{proof}
    \leanok
    \uses{lemma:asymp_ge_nonpos_mul}
    Let $k\_1$ and $k\_2$ be the constants such that $f\_1$ is asymptotically greaterthan
    $g\_1$ and $f\_2$ is asymptotically greater than $g\_2$. By multiplicativity of asymptotic 
    inequality, $f\_1 * f\_2$ is asymptotically less than $k\_1 \cdot g\_1 * k\_2 \cdot g\_2$,
    which is equivalent to asymptotic below boundedness of $f\_1 * f\_2$ by $g\_1 * g\_2$.
\end{proof}
