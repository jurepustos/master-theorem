\section{The Master theorem}

Analyzing the time complexity of algorithms, especially recursive ones, is more often 
than not a non-trivial task. For a recursive algorithm, its time complexity can be 
written as a recurrence formula, which is generally not easy, sometimes even impossible 
to solve with a closed formula. In some cases, though, it turns out that we can find 
asymptotic bounds of the solution, despite not being able to necessarily find the 
precise solution to the recurrence. One class of such cases is the class of 
divide-and-conquer algorithms, i.e. algorithms that work by recursively splitting the input 
problem into similarly-sized subproblems. Those have especially nice recurrence formulas 
which can be asymptotically bounded with an elegant formula that is known as the Master theorem.

\begin{theorem}[Master theorem for divide-and-conquer recurrences]
TODO
\end{theorem}

The full statement of the Master theorem is a bit stronger than the one presented
and proved here. Specifically, the second and third case can both be extended
to a tight asymptotic bound, rather than a just an upper bound.
