\section{The Master theorem}

Analyzing the time complexity of algorithms, especially recursive ones, is more often 
than not a non-trivial task. For a recursive algorithm, its time complexity can be 
written as a recurrence formula, which is generally not easy, sometimes even impossible 
to solve with a closed formula. In some cases, though, we can find asymptotic bounds 
of the solution, despite not being able to necessarily find the 
precise solution to the recurrence. One large class of such cases is the class of 
divide-and-conquer algorithms, i.e. algorithms that recursively split the
problem into smaller, similarly-sized subproblems. The Master theorem puts asymptotic 
bounds on divide-and-conquer recurrences.

\begin{theorem}[Master theorem for divide-and-conquer recurrences]
    Let $T f : \mathbb{N} \rightarrow \mathbb{N}$ be functions such that the recurrence
    \[
        T(n) = a T(n/b) + f(n)
    \]
    holds for some $a > 0$ and $b > 1$. Let also $f(n) \in \Theta(n^d)$ for some $d \ge 1$. 
    Then the following holds:
    \begin{enumerate}
        \item if $a < b^d$, then $T(n) \in \Theta(n^d)$,
        \item if $a = b^d$, then $T(n) \in \Theta(n^d \log_b{n})$ and
        \item if $a > b^d$, then $T(n) \in \Theta(n^{\log_b{a}})$.
    \end{enumerate}
    In the case where $f$ is bounded above by $n^d$ only above or only below,
    $T$ is also bounded only above or only below by the respective function.
\end{theorem}

We prove this theorem by proving all of its cases separately. 

\begin{definition}
    \label{def:upper_master_rec}
    \lean{UpperMasterRec}
    \leanok
    \uses{def:big_o}
    Let $T f : \mathbb{N} \rightarrow \mathbb{N}$ be functions such that the recurrence
    \[
        T(n) \leq a T(\lceil n/b \rceil) + f(n)
    \]
    holds for all $n \ge n\_0$ for some $a > 0$, $n\_0 > 1$ and $b > n\_0$. 
    Let also $f(n) \in O(n^d)$ for some $d \ge 1$. The above recurrence 
    is an \textbf{upper master recurrence} with parameters $(a, n\_0, b, d)$.
\end{definition}

\begin{definition}
    \label{def:lower_master_rec}
    \lean{LowerMasterRec}
    \leanok
    \uses{def:big_omega}
    Let $T f : \mathbb{N} \rightarrow \mathbb{N}$ be functions such that the recurrence
    \[
        T(n) \geq a T(\lfloor n/b \rfloor) + f(n)
    \]
    holds for all $n \ge n\_0$ for some $a > 0$, $n\_0 > 1$ and $b > n\_0$. 
    Let also $f(n) \in \Omega(n^d)$ for some $d \ge 1$. The above recurrence 
    is a \textbf{lower master recurrence} with parameters $(a, n\_0, b, d)$.
\end{definition}

\begin{lemma}
    \label{lemma:big_o_geom}
    \lean{LowerMasterRec.O_geom}
    \leanok
    \uses{def:lower_master_rec, def:big_o}
    Let $T$ and $f$ be functions that form a lower master recurrence with 
    parameters $(a, n\_0, b, d)$. Then 
    $T(n) \in O(n^{\log_b{a}} + \sum_{k=0}^{\lfloor \log_b{n} \rfloor} (\frac{a}{b^d})^k n^d)$.
\end{lemma}

\begin{proof}
    \notready
    TODO
\end{proof}

\begin{theorem}
    \label{thm:lower_master_rec_big_o_of_lt}
    \lean{LowerMasterRec.O_of_lt}
    \notready
    \uses{def:lower_master_rec, def:big_o}
    TODO
\end{theorem}

\begin{proof}
    \notready
    TODO
\end{proof}

\begin{theorem}
    \label{thm:lower_master_rec_big_o_of_eq}
    \lean{LowerMasterRec.O_of_eq}
    \notready
    \uses{def:lower_master_rec, def:big_o}
    TODO
\end{theorem}

\begin{proof}
    \notready
    TODO
\end{proof}

\begin{theorem}
    \label{thm:lower_master_rec_big_o_of_gt}
    \lean{LowerMasterRec.O_of_gt}
    \notready
    \uses{def:lower_master_rec, def:big_o}
    TODO
\end{theorem}

\begin{proof}
    \notready
    TODO
\end{proof}

\begin{lemma}
    \label{lemma:big_omega_geom}
    \lean{UpperMasterRec.Ω_geom}
    \leanok
    \uses{def:upper_master_rec, def:big_omega}
    Let $T$ and $f$ be functions that form an upper master recurrence with 
    parameters $(a, n\_0, b, d)$. Then 
    $T(n) \in \Omega(n^{\log_b{a}} + \sum_{k=0}^{\lfloor \log_b{n} \rfloor} (\frac{a}{b^d})^k n^d)$.
\end{lemma}

\begin{proof}
    \notready
    TODO
\end{proof}

\begin{theorem}
    \label{thm:upper_master_rec_big_omega}
    \lean{UpperMasterRec.Ω_pow_d}
    \notready
    \uses{def:upper_master_rec, def:big_omega}
    TODO
\end{theorem}

\begin{proof}
    \notready
    TODO
\end{proof}

\begin{theorem}
    \label{thm:upper_master_rec_big_omega_of_eq}
    \lean{UpperMasterRec.Ω_of_eq}
    \notready
    \uses{def:upper_master_rec, def:big_omega}
    TODO
\end{theorem}

\begin{proof}
    \notready
    TODO
\end{proof}

\begin{theorem}
    \label{thm:upper_master_rec_big_omega_of_gt}
    \lean{UpperMasterRec.Ω_of_gt}
    \notready
    \uses{def:upper_master_rec, def:big_omega}
    TODO
\end{theorem}

\begin{proof}
    \notready
    TODO
\end{proof}


\begin{corollary}
    \label{thm:master_rec_theta_of_lt}
    \lean{MasterRec.Θ_of_lt}
    \notready
    \uses{def:upper_master_rec, def:lower_master_rec, def:theta}
    TODO
\end{corollary}

\begin{proof}
    \notready
    TODO
\end{proof}

\begin{corollary}
    \label{thm:master_rec_theta_of_eq}
    \lean{MasterRec.Θ_of_eq}
    \notready
    \uses{def:upper_master_rec, def:lower_master_rec, def:theta}
    TODO
\end{corollary}

\begin{proof}
    \notready
    TODO
\end{proof}

\begin{corollary}
    \label{thm:master_rec_theta_of_gt}
    \lean{MasterRec.Θ_of_gt}
    \notready
    \uses{def:upper_master_rec, def:lower_master_rec, def:theta}
    TODO
\end{corollary}

\begin{proof}
    \notready
    TODO
\end{proof}
